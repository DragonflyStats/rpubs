	\documentclass[a4paper,12pt]{article}
%%%%%%%%%%%%%%%%%%%%%%%%%%%%%%%%%%%%%%%%%%%%%%%%%%%%%%%%%%%%%%%%%%%%%%%%%%%%%%%%%%%%%%%%%%%%%%%%%%%%%%%%%%%%%%%%%%%%%%%%%%%%%%%%%%%%%%%%%%%%%%%%%%%%%%%%%%%%%%%%%%%%%%%%%%%%%%%%%%%%%%%%%%%%%%%%%%%%%%%%%%%%%%%%%%%%%%%%%%%%%%%%%%%%%%%%%%%%%%%%%%%%%%%%%%%%
\usepackage{eurosym}
\usepackage{vmargin}
\usepackage{amsmath}
\usepackage{graphics}
\usepackage{epsfig}
\usepackage{subfigure}
\usepackage{enumerate}
\usepackage{fancyhdr}

\setcounter{MaxMatrixCols}{10}
%TCIDATA{OutputFilter=LATEX.DLL}
%TCIDATA{Version=5.00.0.2570}
%TCIDATA{<META NAME="SaveForMode"CONTENT="1">}
%TCIDATA{LastRevised=Wednesday, February 23, 201113:24:34}
%TCIDATA{<META NAME="GraphicsSave" CONTENT="32">}
%TCIDATA{Language=American English}

\pagestyle{fancy}
\setmarginsrb{20mm}{0mm}{20mm}{25mm}{12mm}{11mm}{0mm}{11mm}
\lhead{MA4128} \rhead{Kevin O'Brien} \chead{Feature Engineering} %\input{tcilatex}

\begin{document}

\section*{Dummy variables}
\begin{itemize}
\item When an explanatory variable is categorical we can use \textbf{dummy variables} to contrast
the different categories. 
\item For each variable we choose a \textbf{baseline category} and then
contrast all remaining categories with the base line.
\item If an explanatory variable
has $k$ categories, we need $k-1$ dummy variables to investigate all the differences in
the categories with respect to the dependent variable.
\end{itemize}

For example suppose the explanatory variable was \textbf{\textit{housing}} coded like this:
\begin{itemize}
	\item[1:] Owner occupier
	\item[2:] renting from a private landlord
	\item[3:] renting from the local authority
\end{itemize}

\noindent We would therefore need to choose a baseline category and create two dummy
variables. For example if we chose owner occupier as the baseline category we
would code the dummy variables (\texttt{House1} and \texttt{House2}) like this

\begin{center}
\begin{tabular}{|l|c|c|} \hline 
	Tenure: &House1 &House2\\ \hline \hline
	Owner occupier &0& 0\\ \hline
	Rented private &1 &0\\ \hline
	Rented local authority &0 &1\\ \hline
\end{tabular}

\end{center}

\end{document}
\newpage
\section*{One-Hot Encoding}
One hot encoding is a process by which categorical variables are converted into a form that could be provided to ML algorithms to do a better job in prediction.
\begin{verbatim}
╔════════════╦═════════════════╦════════╗ 
║ CompanyName Categoricalvalue ║ Price  ║
╠════════════╬═════════════════╣════════║ 
║ VW         ╬      1          ║ 20000  ║
║ Acura      ╬      2          ║ 10011  ║
║ Honda      ╬      3          ║ 50000  ║
║ Honda      ╬      3          ║ 10000  ║
╚════════════╩═════════════════╩════════╝
\end{verbatim}

One hot encoding is a process by which categorical variables are converted into a form that could be provided to ML algorithms to do a better job in prediction.

Example: Suppose you have ‘flower’ feature which can take values ‘daffodil’, ‘lily’, and ‘rose’. One hot encoding converts ‘flower’ feature to three features, ‘is\_daffodil’, ‘is\_lily’, and ‘is\_rose’ which all are binary.
\end{document}