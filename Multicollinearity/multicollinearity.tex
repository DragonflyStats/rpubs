
\documentclass[a4paper,12pt]{article}
%%%%%%%%%%%%%%%%%%%%%%%%%%%%%%%%%%%%%%%%%%%%%%%%%%%%%%%%%%%%%%%%%%%%%%%%%%%%%%%%%%%%%%%%%%%%%%%%%%%%%%%%%%%%%%%%%%%%%%%%%%%%%%%%%%%%%%%%%%%%%%%%%%%%%%%%%%%%%%%%%%%%%%%%%%%%%%%%%%%%%%%%%%%%%%%%%%%%%%%%%%%%%%%%%%%%%%%%%%%%%%%%%%%%%%%%%%%%%%%%%%%%%%%%%%%%
\usepackage{eurosym}
\usepackage{vmargin}
\usepackage{amsmath}
\usepackage{graphics}
\usepackage{epsfig}
\usepackage{framed}
\usepackage{subfigure}
\usepackage{fancyhdr}

\setcounter{MaxMatrixCols}{10}
%TCIDATA{OutputFilter=LATEX.DLL}
%TCIDATA{Version=5.00.0.2570}
%TCIDATA{<META NAME="SaveForMode"CONTENT="1">}
%TCIDATA{LastRevised=Wednesday, February 23, 201113:24:34}
%TCIDATA{<META NAME="GraphicsSave" CONTENT="32">}
%TCIDATA{Language=American English}

\pagestyle{fancy}
\setmarginsrb{20mm}{0mm}{20mm}{25mm}{12mm}{11mm}{0mm}{11mm}
\lhead{MA4128} \rhead{Kevin O'Brien} \chead{Multicollinearity} %\input{tcilatex}

%http://www.electronics.dit.ie/staff/ysemenova/Opto2/CO_IntroLab.pdf
\begin{document}

\section{Multicollinearity}
\subsection{Overfitting}
Overfitting occurs when a statistical model does not adequately describe of the underlying
relationship between variables in a regression model. Overfitting generally occurs when the
model is excessively complex, such as having too many parameters (i.e. predictor variables)
relative to the number of observations. A model which has been overfit will generally have poor
predictive performance, as it can exaggerate minor fluctuations in the data.





\subsection{Multicollinearity}
\begin{itemize}
	\item In multiple regression, two or more predictor variables are colinear if they show strong
	linear relationships. This makes estimation of regression coefficients impossible. It can
	also produce unexpectedly large estimated standard errors for the coefficients of the X
	variables involved.
	\item Multicollinearity occurs when two or more predictors in the model are correlated
	and provide redundant information about the response.
	\item This is why an exploratory analysis of the data should be first done to see if any collinearity
	among explanatory variables exists.
	\item Multicolinearity is suggested by non-significant results in individual tests on the regression
	coefficients for important explanatory (predictor) variables.
	\item Multicolinearity may make the determination of the main predictor variable having an
	effect on the outcome difficult.
	\item When choosing a predictor variable you should select one that might be correlated with
	the criterion variable, but that is not strongly correlated with the other predictor variables.
	\item  Examples of pairs of multicollinear predictors are years of education and income, height and weight of a
	person, and assessed value and square footage of a house.
	\item However, correlations amongst the predictor variables are not unusual. The term multi-
	collinearity is used to describe the situation when a high correlation is detected between
	two or more predictor variables.
	\item Such high correlations cause problems when trying to draw inferences about the relative
	contribution of each predictor variable to the success of the model.
\end{itemize}




%=======================%
\subsection{Types of multicollinearity}
There are two types of multicollinearity:
\begin{enumerate}
\item Structural multicollinearity
\item Data-based multicollinearity
\end{enumerate}
Structural multicollinearity is a mathematical artifact caused by creating new predictors from
other predictors such as, creating the predictor x2 from the predictor x. Data-based multi-
collinearity, on the other hand, is a result of a poorly designed experiment, reliance on purely
observational data, or the inability to manipulate the system on which the data are collected.
In the case of structural multicollinearity, the multicollinearity is induced by what you have
done. Data-based multicollinearity is the more troublesome of the two types of multicollinearity.
Unfortunately it is the type we encounter most often!

\subsection{Consequences of high multicollinearity:}
Multicollinearity leads to decreased reliability and predictive power of statistical models, and hence, very
often, confusing and misleading results.

\begin{itemize}
	\item Increased standard error of estimates of the regression coefficients (i.e. decreased reliability of fitted
	model).
	\item Often confusing and misleading results.
	\item Multicollinearity will be dealt with in a future component of this course: Variable Selection Procedures.
	\item This issue is not a serious one with respect to the
	usefulness of the overall model, but it does affect any attempt to interpret the meaning of the partial regression
	coefficients in the model.
%	\item  Multicollinearity will be dealt with in a future component of this course: Variable Selection Procedures.
	\item  This issue is not a serious one with respect to the usefulness of the overall model, but it does affect any attempt to interpret the meaning of the partial regression coefficients in the model.

\end{itemize}

\subsection{Effects of Multicollinearity}

In statistics, the occurrence of several independent variables in a multiple regression model are
closely correlated to one another. Multicollinearity can cause strange results when attempting
to study how well individual independent variables contribute to an understanding of the de-
pendent variable. In general, multicollinearity can cause wide confidence intervals and strange
p−values for independent variables.

%%%%%%%%%%%%%%%%%%%%%%%%%%%%%%%%%%%%%%%%%%%%%%%%%%%%%%%%%%%%%%%%%%%%%%%%%%%%%%%%%
\bigskip

You can also assess multicollinearity in regression in the following ways:

\begin{itemize}
\item [(1)] Examine the correlations and associations (nominal variables) between independent variables to detect a high level of association. High bivariate correlations are easy to spot by running correlations among your variables. If high bivariate correlations are present, you can delete one of the two variables. However, this may not always be sufficient.

\item [(2)] Regression coefficients will change dramatically according to whether other variables are included or excluded from the model. Play around with this by adding and then removing variables from your regression model.

\item [(3)] The standard errors of the regression coefficients will be large if multicollinearity is an issue.

\item [(4)] Predictor variables with known, strong relationships to the outcome variable will not achieve statistical significance. In this case, neither may contribute significantly to the model after the other one is included. But together they contribute a lot. If you remove both variables from the model, the fit would be much worse. So the overall model fits the data well, but neither X variable makes a significant contribution when it is added to your model last. When this happens, multicollinearity may be present.

\end{itemize}


\subsection{How to Identify Multicollinearity}
\begin{itemize}
\item You can assess multicollinearity by examining two collinearity diagnostic measures: tol-
erance and the Variance Inflation Factor (VIF) .
\item Tolerance is a measure of collinearity reported by most statistical programs such as SPSS;
the variables tolerance is $1 - R^2$ .
\item All variables involved in the linear relationship will have a small tolerance.

\item The variance inflation factor (VIF) quantifies the severity of multicollinearity in a regres-
sion analysis.
\item The VIF provides an index that measures how much the variance (the square of the
estimate’s standard deviation) of an estimated regression coefficient is increased because
of collinearity.


\item  You should consider the options to break up the multicollinearity: collecting additional data, deleting predictors, using different predictors, or an alternative to least square regression.

\end{itemize}
\subsection*{The Variance Inflation Factor (VIF)}
\begin{itemize}
\item The Variance Inflation Factor (VIF) measures the impact of collinearity among the variables in a regression model.
\item The Variance Inflation Factor (VIF) is 1/Tolerance, it is always greater than or equal to
1.
\item There is no formal VIF value for determining presence of multicollinearity. Values of VIF
that exceed 10 are often regarded as indicating multicollinearity, but in weaker models
values above 2.5 may be a cause for concern.
\item In many statistics programs, the results are shown both as an individual $R^2$ value (distinct
from the overall $R^2$ of the model) and a Variance Inflation Factor (VIF).
\item When those $R^2$ and VIF values are high for any of the variables in your model, multi-
collinearity is probably an issue.

\item When VIF is high there is high multicollinearity and instability of the b and beta coefficients. It is often difficult to sort this out. 




\end{itemize}


%===============================================%

\section{Interpreting VIF Multicollinearity}

\begin{itemize}

\item A common rule of thumb is that if the VIF is greater than 5 then multicollinearity is high. Also a VIF level of 10 has been proposed as a cut off value.

\item  The variance inflation factor (VIF) is used to detect whether one predictor has a strong linear association
with the remaining predictors (the presence of multicollinearity among the predictors).

\item  VIF measures how much the variance of an estimated regression coefficient increases if your predictors
are correlated (multicollinear). VIF = 1 indicates no relation; VIF > 1, otherwise.

\item  The largest VIF among all predictors is often used as an indicator of severe multicollinearity.
\item  \textbf{\textit{Montgomery and Peck}} suggest that when VIF is greater than 5-10, then the regression coefficients are poorly estimated.
\item A common rule of thumb is that if the VIF is greater than 5 then multicollinearity is high.
Also a VIF level of 10 has been proposed as a cut off value.

\end{itemize}

%%%%%%%%%%%%%%%%%%%%%%%%%%%%%%%%%%%%%%%%%%%%%%%%%%%%%%%%%%%%%%%%%%%%%%%%%

%
%
%\subsection{Interpreting Variance Inflation Factors}
%\begin{itemize}
%	\item We learned previously that the standard errors, and hence the variances, of the estimated
%	coefficients are inflated when multicollinearity exists.
%	
%	
%	\item So, the variance inflation factor for the estimated coefficient b k , denoted $VIF_k$ , is just the
%	factor by which the variance is inflated.
%	\item Variance inflation factors k greater than 4 suggest that the multicollinearity should be
%	investigated.
%	\item Variance inflation factors greater than 10 are taken as an indication that the multicollinearity may be unduly influencing the least squares estimates.
%\end{itemize}








\newpage
\subsection{Determing the Variance Inflation Factor (VIF) with R}
\begin{verbatim}
library(car)
# Evaluate Collinearity
vif(fit) # variance inflation factors
sqrt(vif(fit)) > 2 # problem?
\end{verbatim}
%
%\subsection{The Variance Inflation Factor (VIF)}
%
%    In many statistics programs, the results are shown both as an individual R2 value (distinct from the overall R2 of the model) and a Variance Inflation Factor (VIF). When those R2 and VIF values are high for any of the variables in your model, multicollinearity is probably an issue. When VIF is high there is high multicollinearity and instability of the b and beta coefficients. It is often difficult to sort this out.


%
%\subsection{Multicollinearity}
%When choosing a predictor variable you should select one that might be correlated with the criterion variable, but that is not strongly correlated with the other predictor variables. However, correlations amongst the predictor variables are not unusual. The term multicollinearity (or collinearity) is used to describe the situation
%when a high correlation is detected between two or more predictor variables.
%
%Such high correlations cause problems when trying to draw inferences about the relative contribution of each predictor variable to the success of the model. SPSS provides you with a means of checking for this and we describe this below.
%
%\subsection{Variance Inflation Factor (VIF)}
%
%The variance inflation factor (or “VIF”) provides us with a measure of how much the variance for a given regression coefficient is increased compared to if all predictors were uncorrelated. To understand what the variance inflation factor is, and what it measures, we need to examine the computation of the standard error of a regression coefficient.
%



%The variance inflation factor (or ``VIF") provides us with a measure of how much the variance for a given regression coefficient is increased compared to if all predictors were uncorrelated. To understand what the variance inflation factor is, and what it measures, we need to examine the computation of the standard error of a regression coefficient.






%%%%%%%%%%%%%%%%%%%%%%%%%%%%%%%%%%%%%%%%%%%%%%%%%%%%%%%%%%%%%%%%%%%%%%%%%%%%%%%%%%%%%%%%%%%%%%%%%%%%%
\subsection*{Tolerance}
Tolerance is simply the reciprocal of VIF, and is computed as


\[ Tolerance = \frac{1}{VIF}\]

Whereas large values of VIF were unwanted and undesirable, since tolerance is the reciprocal
of VIF, larger than not values of tolerance are indicative of a lesser problem with collinearity.
In other words, we want large tolerances.

\begin{itemize}
\item A tolerance close to 1 means there is little multicollinearity, whereas a value close to 0
suggests that multicollinearity may be a threat.
\item The VIF shows us how much the variance of the coefficient estimate is being inflated by
multicollinearity. For example, if the VIF for a variable were 9, its standard error would
be three times as large as it would be if its VIF was 1. In such a case, the coefficient
would have to be 3 times as large to be statistically significant.
\item Interpretation: A small tolerance value indicates that the variable under consideration
is almost a perfect linear combination of the independent variables already in the equation
and that it should not be added to the regression equation.
\item Interpretation: Some suggest that a tolerance value less than 0.1 should be investigated
further. If a low tolerance value is accompanied by large standard errors and nonsignifi-
cance, multicollinearity may be an issue.
\end{itemize}



\end{document}
